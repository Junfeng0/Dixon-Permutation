\documentclass[12pt]{amsart}

\topmargin=-0.4in \oddsidemargin=0.2in \evensidemargin=0.2in
\textwidth=6.2in \textheight=9in

\usepackage{amssymb,amsfonts,amsmath,amsthm,epsfig,tikz,bm,color}
%\usepackage{showkeys}
%=================================================
\def\dfrac{\displaystyle\frac} \def\ovl{\overline} \def\Int{\displaystyle\int}
\def\Rm#1{\lowercase\expandafter{\romannumeral#1}}
\def\for{\forall~} \def\exi{\exists~} \def\c{\subseteq}
\def\iif{\Leftrightarrow} \def\Rto{\Rightarrow} \def\Lto{\Leftarrow}
\def\oo{\infty} \def\pa{\partial} \def\rto{\rightarrow} \def\mto{\mapsto}
\def\ug{\triangle} \def\dg{\nabla} \def\lg{\langle} \def\rg{\rangle}\def\Ga{\Gamma}\def\Core{{\sf Core}}
%=================================================
\def\T{\Gamma}  \def\D{\Delta} \def\Th{\Theta}
\def\Lmd{\Lambda} \def\E{\Sigma} \def\O{\Omega}
\def\a{\alpha} \def\b{\beta} \def\g{\gamma} \def\d{\delta} \def\e{\varepsilon}
\def\r{\rho} \def\o{\sigma} \def\t{\tau} \def\w{\omega} \def\k{\kappa}
\def\th{\theta} \def\lmd{\lambda} \def\ph{\varphi} \def\z{\zeta} \def\ti{\tilde} \def\SS{{\rm S}}\def\DD{{\rm D}}
%=================================================
\def\A{$A$} \def\G{$G$} \def\H{$H$} \def\K{$K$} \def\M{$M$} \def\N{$N$}
\def\P{$P$} \def\Q{$Q$} \def\R{$R$} \def\V{$V$} \def\X{$X$} \def\Y{$Y$}
\def\bfA{{\bf A}} \def\rmD{{\rm D}} \def\bfS{{\bf S}} \def\bfK{{\bf K}}
\def\bfM{{\bf M}} \def\bbZ{{\Bbb Z}} \def\GL{{\bf GL}} \def\C{Cayley\,}
\def\CS{Cay(G,S)} \def\CT{Cay(G,T)} \def\Iso{Iso(S,T)} \def\B{$B$}\def\ZZ{\mathbb Z}
%=================================================
\def\oa{\ovl A} \def\og{\ovl G} \def\oh{\ovl H} \def\ob{\ovl B} \def\oq{\ovl Q}
\def\oc{\ovl C} \def\ok{\ovl K} \def\ol{\ovl L} \def\om{\ovl M} \def\on{\ovl N}
\def\op{\ovl P} \def\oR{\ovl R} \def\os{\ovl S} \def\ot{\ovl T} \def\ou{\ovl U}
\def\ov{\ovl V} \def\ow{\ovl W} \def\ox{\ovl X} \def\oT{\ovl\T}
%=================================================
\def\di{\bigm|} \def\Di{\Bigm|} \def\nd{\mathrel{\bigm|\kern-.7em/}}
\def\Nd{\mathrel{\not\,\Bigm|}} \def\edi{\bigm|\bigm|}
\def\m{\medskip} \def\l{\noindent} \def\x{$\!\,$} \def\J{$-\!\,$}
%=================================================
\def\Hom{\hbox{\rm Hom}} \def\Aut{\hbox{\rm Aut}} \def\Inn{\hbox{\rm Inn}}  \def\OD{\hbox{\rm OD}}
\def\Syl{\hbox{\rm Syl}} \def\Sym{\hbox{\rm Sym}} \def\Alt{\hbox{\rm Alt}}
\def\Ker{\hbox{\rm Ker}} \def\fix{\hbox{\rm fix}} \def\mod{\hbox{\rm mod}\,}
\def\GL{\hbox{\rm GL}} \def\GF{\hbox{\rm GF}} \def\PG{\hbox{\rm PG}}
\def\pgl{\hbox{\bf PGL}} \def\psl{{\bf PSL}} \def\psu{{\bf PSU}}
\def\Cay{\hbox{\rm Cay}} \def\Mult{\hbox{\rm Mult}} \def\Iso{\hbox{\rm Iso}}
\def\Val{\hbox{\rm Val}} \def\Sab{\hbox{\rm Sab}} \def\supp{\hbox{\rm supp}}
\def\qed{\hfill $\Box$} \def\qqed{\qed\vspace{3truemm}}
\def\CS{\Cay(G,S)} \def\CT{\Cay(G,T)} \def\Arc{\hbox{\rm Arc}} \def\Cos{{\rm Cos}}
\def\fs{\footnotesize} \def\bbZ{{\Bbb Z}} \def\bbQ{{\Bbb Q}} \def\bbN{{\Bbb N}}
\def\Hol{\hbox{\rm Hol}} \def\rad{\hbox{\rm rad}}
%=================================================

\newtheorem{thm}{Theorem}[section]
\newtheorem{cor}[thm]{Corollary}
\newtheorem{lem}[thm]{Lemma}
\def\theequation{\thesection.\arabic{equation}}
\makeatletter \@addtoreset{equation}{section}
\newtheorem{rmk}{Remark}[section]
\newtheorem*{rmk1}{Remark}
\newtheorem{que}[thm]{Question}
\newtheorem{pro}[thm]{Proposition}
\newtheorem{con}[thm]{Condition}
\newtheorem{rem}[thm]{\it Remarks}
\newtheorem{defi}[thm]{Definition}
\newtheorem{exa}{\it Example}
\newtheorem{obs}{\it Observation}
\newtheorem{prob}[thm]{Problem}
\def\pf{\noindent {\it Proof.\ }}
\def\qed{\ifmmode\square\else\nolinebreak\hfill
$\Box$\fi\par\vskip12pt}






\begin{document}


\title[Group Actions]
{Group Actions}%
\maketitle

\l Exercises

\l 1. Let $\r: G\mapsto Sym(\O)$ be a representation of the group $G$ on the set $\O$. Show that this defines an action of $G$ on $\O$ by setting $\a^x:=\a^{\r(x)}$ for all $\a\in \O$ and $x\in G$, and that $\r$ is the representation which corresponds to this action.

\pf Since $\r$ is a representation of $G$ on $\O$, $\r$ is a homomorphism. Let $\a^x:=\a^{\r(x)}$, then
\begin{align*}
 &\a^1=\a^{\r(1)}=\a;\\
 &(\a^x)^y=(\a^{\r(x)})^{\r(y)}=\a^{\r(xy)}=\a^{xy}.
\end{align*}
Thus, this defines an action of $G$ on $\O$.

\m

\l 2. Explain why we do not usually get an action of a group $G$ on itself by defining $a^x:=xa$. Show, however, that $a^x:=x^{-1}a$ does give an action of $G$ on itself (called the left regular representation of $G$). Similarly, show how to define an action of a group on the set of left cosets $aH$($a\in G$) of a subgroup $H$.

\pf Let $a^x:=xa$, then
\begin{align*}
 (a^x)^y=(xa)^y=yxa \neq xya=a^{xy}.
\end{align*}
The equation holds if and only if $G$ is an abelian group.

For all $a\in G$ and $x\in G$, let $a^x:=x^{-1}a$, then
\begin{align*}
 &a^1=1^{-1}a=a;\\
 &(a^x)^y=(x^{-1}a)^{y}=y^{-1}x^{-1}a=(xy)^{-1}a=a^{xy}.
\end{align*}
Thus, this defines an action of $G$ on itself.

Let $\O=\{aH|\ a\in G\}$, defining an action of $G$ on $\O$ by setting $(aH)^x:=x^{-1}aH$.

\m

\l 3. Show that the kernel of $\r_H$ in Example 1.3.4 is equal to the largest normal subgroup of $G$ contained in the subgroup $H$.

\pf Let $\Ga_H:=\{Ha|\ a\in G\}$ and define an action if $G$ on $\Ga_H$ by right multiplication: $(Ha)^x:=Hax$. We denote the corresponding representation of $G$ on $\Ga_H$ by $\r_H$. We have
\begin{align*}
 {\rm ker}\ \r_H\ =\ \bigcap_{a\in G}a^{-1}Ha.
\end{align*}
Assume that $N$ is the subgroup of $H$ and $N\unlhd G$, then we have $N=N^a\le H^a$ for any $a\in G$. Thus, we obtained that $N\le \bigcap_{a\in G}H^a={\rm ker}\ \r_H$.

Hence, by the arbitrariness of $N$, ${\rm ker}\ \r_H$ is equal to the largest normal subgroup of $G$ contained in the subgroup $H$.

\m

\l 4. Use the previous exercise to prove that if $G$ is a group with a subgroup $H$ of finite index $n$, then $G$ has a normal subgroup $K$ contained in $H$ whose index in $G$ is finite and divides $n!$. In particular, if $H$ has index 2 then $H$ is normal in $G$.

\pf By ex.3, we have that ${\rm ker}\ \r_H$ is the largest normal subgroup of $G$ contained in the subgroup $H$. 
Let $K= {\rm ker}\ \r_H$.
Since $\r_H$ is the action of $G$ on $\Ga_H$ by right multiplication: $(Ha)^x:=Hax$ and $|G:H|=n$, we have that $G/ {\rm ker}\ \r_H \lesssim S_n$, that is, $|G:K|\mid n!$.
Hence, $|G:K|$ is finite and divides $n!$.

If $n=2$, then $|G:{\rm ker}\ \r_H|$=1 or 2. If $|G:{\rm ker}\ \r_H|=1$, then $G={\rm ker}\ \r_H=\bigcap_{a\in G}a^{-1}Ha$ which implies that $G=H$, contradiction. Thus, $|G:{\rm ker}\ \r_H|=2$, then $H={\rm ker}\ \r_H$, that is, $H$ is normal in $G$.

\m

\l 5. Let $G$ be a finite group, and let $p$ be the smallest prime which divides the order of $G$. If $G$ has a subgroup $H$ of index $p$, show that $H$ must be normal in $G$. In particular, in a finite $p$-group(that is, a group of order $p^k$ for some prime $p$) any subgroup of index $p$ is normal.




\end{document}
